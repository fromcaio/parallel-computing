%% ===================================================================================
%% Elementos Pré-Textuais (pretextual_elements.tex)
%% Versão: 2.2 (Corrigida)
%%
%% Este arquivo contém:
%% - Folha de Aprovação
%% - Dedicatória
%% - Agradecimentos
%% - Epígrafe
%% - Resumo em Português
%% - Resumo em Língua Estrangeira (ex: Inglês)
%%
%% Preencha, descomente ou remova as seções conforme sua necessidade.
%% ===================================================================================


% -----------------------------------------------------------------------------------
% -- FOLHA DE APROVAÇÃO
% -----------------------------------------------------------------------------------
% NOTA: Este é um elemento obrigatório (ABNT NBR 14724:2011).
% Você pode usar esta versão em texto para rascunhos. Após a defesa, é uma
% prática comum escanear a página assinada e substituir todo o conteúdo do
% ambiente 'folhadeaprovacao' por: \includegraphics[width=\textwidth]{caminho/para/pagina_assinada.pdf}
% -----------------------------------------------------------------------------------
% \begin{folhadeaprovacao}
%   \begin{center}
%     {\ABNTEXchapterfont\large\imprimirautor}

%     \vspace*{\fill}
%     {\ABNTEXchapterfont\bfseries\Large\imprimirtitulo}
%     \vspace*{\fill}

%     \hspace{.45\textwidth}
%     \begin{minipage}{.5\textwidth}
%       \SingleSpacing
%       \imprimirpreambulo
%     \end{minipage}%
%     \vspace*{\fill}
%   \end{center}

%   % [A FAZER: Ajuste a data de aprovação. \today usa a data da compilação.]
%   Trabalho aprovado. \imprimirlocal, \today:
%   \vspace*{2cm}

%   % [A FAZER: Preencha com os nomes dos membros da sua banca examinadora.]
%   \assinatura{\textbf{\imprimirorientador} \\ Orientador(a)}
%   \assinatura{\textbf{Nome Completo Membro da Banca 1} \\ Instituição}
%   \assinatura{\textbf{Nome Completo Membro da Banca 2} \\ Instituição}
%   % \assinatura{\textbf{Nome Completo do Membro da Banca 3} \\ Instituição} % Descomente se necessário

% \end{folhadeaprovacao}


% -----------------------------------------------------------------------------------
% -- DEDICATÓRIA (Opcional)
% -----------------------------------------------------------------------------------
% [A FAZER: Escreva sua dedicatória aqui ou comente/remova todo o ambiente.]
% \begin{dedicatoria}
% 	\vspace*{\fill} % Centraliza o texto verticalmente
% 	\flushright    % Alinha o texto à direita
% 	\textit{DEDICATÓRIA: Agradeço aos ..., pelo apoio e incentivo incondicional...}
% \end{dedicatoria}


% -----------------------------------------------------------------------------------
% -- AGRADECIMENTOS (Opcional)
% -----------------------------------------------------------------------------------
% [A FAZER: Escreva seus agradecimentos aqui ou comente/remova todo o ambiente.]
% \begin{agradecimentos}
% Agradeço a todos que contribuíram para a realização deste trabalho. Em especial, ao meu orientador, Prof. Dr. [Nome do Orientador], pelo suporte e orientação ao longo de todo o processo.

% \end{agradecimentos}


% -----------------------------------------------------------------------------------
% -- EPÍGRAFE (Opcional)
% -----------------------------------------------------------------------------------
% [A FAZER: Adicione uma epígrafe aqui ou comente/remova todo o ambiente.]
% \begin{epigrafe}
% 	\vspace*{\fill}
% 	\begin{flushright}
% 		\textit{``A educação é a arma mais poderosa que você pode usar para mudar o mundo."} \\
% 		\textendash{} Nelson Mandela
% 	\end{flushright}
% \end{epigrafe}


% -----------------------------------------------------------------------------------
% -- RESUMO EM PORTUGUÊS (Obrigatório)
% -----------------------------------------------------------------------------------
% \begin{resumo}
% 	% [A FAZER: Escreva o resumo do seu trabalho em português aqui.]
% 	O resumo deve apresentar de forma concisa o tema do trabalho, o problema de pesquisa, os objetivos, a metodologia utilizada e os principais resultados ou conclusões alcançadas. Ele é uma parte crucial do trabalho, pois fornece uma visão geral que permite ao leitor compreender rapidamente o escopo e a contribuição da sua pesquisa. O resumo deve ser escrito em um único parágrafo, contendo entre 150 e 500 palavras, conforme a ABNT.

% 	\vspace{\onelineskip} % Adiciona uma linha em branco antes das palavras-chave
% 	\noindent % Impede a indentação da linha das palavras-chave
% 	\textbf{Palavras-chave}: Engenharia de Software; Metodologia Ágil; Qualidade de Software.
% \end{resumo}


% -----------------------------------------------------------------------------------
% -- RESUMO EM LÍNGUA ESTRANGEIRA (Obrigatório)
% -----------------------------------------------------------------------------------
\begin{resumo}[Abstract]
	\begin{otherlanguage*}{english}

		This project presents the implementation and performance analysis of the \textit{Sieve of Eratosthenes} algorithm for prime number generation, developed in the C programming language. The work explores two complementary approaches: a sequential baseline and a parallel version using the Message Passing Interface (MPI) under the Master–Slave computational model. The sequential implementation provides a reference for algorithmic correctness and serves as a benchmark for measuring the efficiency of the parallel solution. The parallel version decomposes the computation range into balanced subintervals distributed among MPI processes, while the master node is responsible for broadcasting base primes, collecting partial results, and consolidating the final output. Execution time was measured using standard timing functions to isolate the computation phase, excluding input/output overhead. The study demonstrates how data decomposition and message passing can significantly reduce execution time for large upper bounds of $N$, reinforcing the suitability of MPI for scalable parallelization of computationally intensive algorithms. The resulting program highlights key concepts of parallel computing, including workload distribution, synchronization, and performance evaluation through speedup and efficiency metrics.

		\vspace{\onelineskip}
		\noindent
		\textbf{Keywords}: Parallel Computing; Message Passing Interface; Sieve of Eratosthenes; Performance Analysis; C Programming.

	\end{otherlanguage*}
\end{resumo}