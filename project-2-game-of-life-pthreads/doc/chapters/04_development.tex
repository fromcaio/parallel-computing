\chapter{Implementation}\label{cap:development}

\section{Data Structures and I/O}

The grid is stored as a contiguous array of bytes to maximize spatial locality. A simple \texttt{Grid} struct holds dimensions and a pointer to the cell buffer. Two buffers (current/next) implement double-buffering to avoid read--write hazards during updates. Input parsing validates dimensions and coordinates, and outputs reuse the same format as the input.

\section{Update Kernel}

The core step iterates over assigned rows, counts eight neighbors, and writes the next state. The kernel is identical for both builds and lives in a shared module. Boundary checks clamp neighbor coordinates; no wrap-around is applied.

\section{Sequential Version}

The sequential binary drives the simulation loop, swapping buffers after each generation. Timing excludes file I/O, and results are written to \texttt{output/game\_of\_life\_seq\_<rows>x<cols>\_<gen>gen.txt}.

\section{Parallel Version (Pthreads)}

The parallel driver partitions rows into nearly equal blocks (handling remainders to balance work). Each thread receives its start/end rows, a shared pointer to the current/next buffers, and two barriers:
\begin{itemize}
    \item Compute barrier: ensures all threads finish writing their slice for generation $g$.
    \item Swap barrier: synchronizes after the master thread swaps \texttt{current}/\texttt{next} pointers.
\end{itemize}

This results in two barrier synchronizations per generation, with no additional locks. Output files include the thread count and grid size: \texttt{output/game\_of\_life\_threads\_<t>t\_<rows>x<cols>\_<gen>gen.txt}.

\section{Error Handling and Logging}

Input validation checks negative dimensions and out-of-bounds coordinates. Failure paths free allocated buffers and report errors to \texttt{stderr}. Runtime logs print timing and output paths for reproducibility.

\section{Visualization Placeholders}

To aid explanation, placeholders are reserved for figures that will be generated later with Python scripts:

\begin{figure}[h]
    \centering
    \fbox{\parbox{0.8\linewidth}{Placeholder: Game of Life evolution sequence (to be generated in Python).}}
    \caption{(Placeholder) Evolution of a glider over several generations.}
    \label{fig:glider-evolution}
\end{figure}

\begin{figure}[h]
    \centering
    \fbox{\parbox{0.8\linewidth}{Placeholder: Cache-friendly memory layout illustration (to be generated in Python).}}
    \caption{(Placeholder) Contiguous row-major storage of the grid.}
    \label{fig:memory-layout}
\end{figure}
