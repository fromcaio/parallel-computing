%% ===================================================================================
%% Elementos Pré-Textuais (pretextual_elements.tex)
%% Versão: 2.2 (Corrigida)
%%
%% Este arquivo contém:
%% - Folha de Aprovação
%% - Dedicatória
%% - Agradecimentos
%% - Epígrafe
%% - Resumo em Português
%% - Resumo em Língua Estrangeira (ex: Inglês)
%%
%% Preencha, descomente ou remova as seções conforme sua necessidade.
%% ===================================================================================


% -----------------------------------------------------------------------------------
% -- FOLHA DE APROVAÇÃO
% -----------------------------------------------------------------------------------
% NOTA: Este é um elemento obrigatório (ABNT NBR 14724:2011).
% Você pode usar esta versão em texto para rascunhos. Após a defesa, é uma
% prática comum escanear a página assinada e substituir todo o conteúdo do
% ambiente 'folhadeaprovacao' por: \includegraphics[width=\textwidth]{caminho/para/pagina_assinada.pdf}
% -----------------------------------------------------------------------------------
% \begin{folhadeaprovacao}
%   \begin{center}
%     {\ABNTEXchapterfont\large\imprimirautor}

%     \vspace*{\fill}
%     {\ABNTEXchapterfont\bfseries\Large\imprimirtitulo}
%     \vspace*{\fill}

%     \hspace{.45\textwidth}
%     \begin{minipage}{.5\textwidth}
%       \SingleSpacing
%       \imprimirpreambulo
%     \end{minipage}%
%     \vspace*{\fill}
%   \end{center}

%   % [A FAZER: Ajuste a data de aprovação. \today usa a data da compilação.]
%   Trabalho aprovado. \imprimirlocal, \today:
%   \vspace*{2cm}

%   % [A FAZER: Preencha com os nomes dos membros da sua banca examinadora.]
%   \assinatura{\textbf{\imprimirorientador} \\ Orientador(a)}
%   \assinatura{\textbf{Nome Completo Membro da Banca 1} \\ Instituição}
%   \assinatura{\textbf{Nome Completo Membro da Banca 2} \\ Instituição}
%   % \assinatura{\textbf{Nome Completo do Membro da Banca 3} \\ Instituição} % Descomente se necessário

% \end{folhadeaprovacao}


% -----------------------------------------------------------------------------------
% -- DEDICATÓRIA (Opcional)
% -----------------------------------------------------------------------------------
% [A FAZER: Escreva sua dedicatória aqui ou comente/remova todo o ambiente.]
% \begin{dedicatoria}
% 	\vspace*{\fill} % Centraliza o texto verticalmente
% 	\flushright    % Alinha o texto à direita
% 	\textit{DEDICATÓRIA: Agradeço aos ..., pelo apoio e incentivo incondicional...}
% \end{dedicatoria}


% -----------------------------------------------------------------------------------
% -- AGRADECIMENTOS (Opcional)
% -----------------------------------------------------------------------------------
% [A FAZER: Escreva seus agradecimentos aqui ou comente/remova todo o ambiente.]
% \begin{agradecimentos}
% Agradeço a todos que contribuíram para a realização deste trabalho. Em especial, ao meu orientador, Prof. Dr. [Nome do Orientador], pelo suporte e orientação ao longo de todo o processo.

% \end{agradecimentos}


% -----------------------------------------------------------------------------------
% -- EPÍGRAFE (Opcional)
% -----------------------------------------------------------------------------------
% [A FAZER: Adicione uma epígrafe aqui ou comente/remova todo o ambiente.]
% \begin{epigrafe}
% 	\vspace*{\fill}
% 	\begin{flushright}
% 		\textit{``A educação é a arma mais poderosa que você pode usar para mudar o mundo."} \\
% 		\textendash{} Nelson Mandela
% 	\end{flushright}
% \end{epigrafe}


% -----------------------------------------------------------------------------------
% -- RESUMO EM PORTUGUÊS (Obrigatório)
% -----------------------------------------------------------------------------------
\begin{resumo}
    Este trabalho apresenta uma implementação sequencial e uma versão paralela (Pthreads) do Jogo da Vida de Conway em C. A decomposição por blocos de linhas distribui o grid entre as threads, enquanto barreiras sincronizam a progressão de cada geração. O código foi estruturado para separar kernel de atualização, entrada/saída e orquestração, permitindo medir apenas o tempo de computação. Serão avaliados tempo de execução, speedup e eficiência em diferentes tamanhos de matrizes e números de threads. Figuras e gráficos serão gerados posteriormente em Python para ilustrar a evolução de padrões e o particionamento do domínio.

    \vspace{\onelineskip}
    \noindent
    \textbf{Palavras-chave}: Computação Paralela; Pthreads; Autômatos Celulares; Jogo da Vida; Desempenho.
\end{resumo}


% -----------------------------------------------------------------------------------
% -- RESUMO EM LÍNGUA ESTRANGEIRA (Obrigatório)
% -----------------------------------------------------------------------------------
\begin{resumo}[Abstract]
    \begin{otherlanguage*}{english}

        This project implements Conway's Game of Life in C with two versions: a sequential baseline and a shared-memory parallel variant using POSIX threads. A row-block decomposition assigns contiguous slices of the grid to each thread, and per-generation barriers maintain correctness while swapping buffers. The codebase separates the update kernel from orchestration and I/O, enabling timing that excludes file writing. We plan to measure execution time, speedup, and efficiency across grid sizes and thread counts. Placeholders are reserved for Python-generated figures illustrating pattern evolution and domain decomposition.

        \vspace{\onelineskip}
        \noindent
        \textbf{Keywords}: Parallel Computing; Pthreads; Cellular Automata; Game of Life; Performance Evaluation.

    \end{otherlanguage*}
\end{resumo}
